\documentclass{article}
\usepackage[utf8]{inputenc}
\usepackage{amsmath}
\usepackage{amsfonts}
\usepackage{amssymb}
\usepackage{graphicx}
\usepackage{hyperref}
\usepackage{enumitem}
\usepackage{booktabs}
\usepackage{geometry}
\geometry{margin=1in}

\title{AgentCodeEval: A Novel Benchmark for Evaluating Long-Context Language Models in Software Development Agent Tasks}
\author{Research Proposal}
\date{\today}

\begin{document}

\maketitle

\begin{abstract}
We propose AgentCodeEval, a comprehensive benchmark for evaluating long-context language models (LLMs) in software development agent scenarios. Unlike existing code evaluation benchmarks that focus on single-function completion or isolated code understanding, AgentCodeEval addresses the critical gap in evaluating LLMs' ability to perform complex, multi-file software development tasks that require understanding large codebases, managing dependencies, and implementing features across multiple sessions. This benchmark targets real-world software development workflows where agents must navigate complex file structures, understand inter-module dependencies, and maintain architectural consistency across large-scale codebases. With 12,000 evaluation instances, AgentCodeEval provides 5x more comprehensive coverage than the largest existing benchmark (SWE-Bench: 2,294 instances).
\end{abstract}

\section{Introduction}

\subsection{Motivation}
Current long-context LLM evaluation benchmarks exhibit significant limitations when assessing software development agent capabilities:

\subsubsection{Existing Benchmark Limitations}
\begin{itemize}
    \item \textbf{HumanEval (164 instances)}: Single-function code completion, no architectural context
    \item \textbf{MBPP (1,000 instances)}: Isolated problem-solving, no codebase integration
    \item \textbf{SWE-Bench (2,294 instances)}: Single-issue resolution, limited agent workflow evaluation
    \item \textbf{L-Eval (2,000+ instances)}: General long-context tasks, no code-specific evaluation
    \item \textbf{ETHIC (1,986 instances)}: High information coverage but not code-focused
\end{itemize}

\subsubsection{Real-World Agent Requirements}
Software development agents must handle complex, interconnected tasks that existing benchmarks cannot evaluate:
\begin{itemize}
    \item \textbf{Multi-file codebase understanding}: Navigate 50+ files with complex dependencies
    \item \textbf{Cross-module dependency management}: Understand and maintain architectural relationships
    \item \textbf{Multi-session development workflows}: Maintain context across 2-5 development sessions
    \item \textbf{Architectural pattern recognition}: Identify and maintain design patterns at scale
    \item \textbf{Integration of new features}: Add functionality while preserving existing architecture
    \item \textbf{Long-context memory retention}: Remember decisions and constraints across 200K+ tokens
\end{itemize}

\subsection{Research Gap}
Despite the critical importance of AI-powered development tools, no existing benchmark comprehensively evaluates long-context LLMs in realistic software development agent scenarios. Current evaluation methods fall short in three key areas:

\begin{enumerate}
    \item \textbf{Scale}: Largest code benchmark (SWE-Bench) has only 2,294 instances
    \item \textbf{Agent Focus}: No benchmark specifically designed for agent workflows
    \item \textbf{Long-Context Integration}: No evaluation of code understanding at 100K+ token scale
\end{enumerate}

\section{AgentCodeEval: Benchmark Design}

\subsection{Core Concept}
AgentCodeEval evaluates LLMs on realistic software development scenarios that span multiple files, require understanding of complex dependencies, and simulate multi-session development workflows. With 12,000 instances across 8 comprehensive task categories, AgentCodeEval provides 5x more extensive coverage than existing benchmarks.

\subsection{Key Design Principles}
\begin{enumerate}
    \item \textbf{Unprecedented Scale}: 12,000 instances—5x larger than SWE-Bench (2,294)
    \item \textbf{Real-world Relevance}: Tasks derived from actual production codebases in The Stack v2
    \item \textbf{Multi-file Complexity}: Each task involves 15-200+ files with authentic dependencies
    \item \textbf{Long-context Requirement}: Tasks require understanding of 10K-200K+ tokens
    \item \textbf{Agent-specific Evaluation}: Focus on multi-step agent workflows, not single-turn completion
    \item \textbf{High Information Coverage}: Extending ETHIC methodology with IC > 0.7 for all tasks
    \item \textbf{Automated Generation Pipeline}: Scalable creation using tree-sitter and LLM APIs
    \item \textbf{Novel Evaluation Metrics}: Agent-specific metrics for architectural understanding and memory retention
\end{enumerate}

\subsection{Benchmark Architecture}
AgentCodeEval consists of three integrated evaluation layers:
\begin{itemize}
    \item \textbf{Task Layer}: 8 categories covering all aspects of software development agents
    \item \textbf{Context Layer}: Progressive complexity from 10K to 200K+ tokens
    \item \textbf{Evaluation Layer}: Multi-dimensional metrics including architectural coherence and memory retention
\end{itemize}

\subsection{Task Categories}

AgentCodeEval comprises 8 comprehensive task categories, each designed to evaluate specific agent capabilities:

\subsubsection{1. Architectural Understanding (1,500 instances)}
Evaluates agents' ability to comprehend and work with large-scale software architecture:
\begin{itemize}
    \item \textbf{Design pattern recognition}: Identify MVC, Observer, Factory patterns across 50+ files
    \item \textbf{Dependency graph analysis}: Understand component relationships in microservice architectures
    \item \textbf{Layered architecture navigation}: Work with presentation, business, and data layers
    \item \textbf{Context}: 15K-80K tokens, real production codebases from The Stack v2
\end{itemize}

\subsubsection{2. Cross-file Refactoring (1,500 instances)}
Tests systematic code restructuring while maintaining functionality:
\begin{itemize}
    \item \textbf{Extract functionality}: Move shared code into new modules across 20+ files
    \item \textbf{Rename operations}: Update class/function names consistently across codebase
    \item \textbf{Package restructuring}: Reorganize module hierarchies while preserving imports
    \item \textbf{Design pattern application}: Implement patterns like Strategy or Command across multiple files
    \item \textbf{Context}: 20K-100K tokens, complex dependency chains
\end{itemize}

\subsubsection{3. Feature Implementation (1,800 instances)}
Evaluates adding new functionality to existing systems:
\begin{itemize}
    \item \textbf{Authentication systems}: Implement OAuth/JWT across web application layers
    \item \textbf{API endpoint creation}: Add REST endpoints with validation, routing, and documentation
    \item \textbf{Database integration}: Implement new schemas while maintaining data consistency
    \item \textbf{Third-party integration}: Add external services (payment, notification, analytics)
    \item \textbf{Context}: 25K-150K tokens, full-stack development scenarios
\end{itemize}

\subsubsection{4. Bug Investigation (1,500 instances)}
Tests systematic debugging across complex codebases:
\begin{itemize}
    \item \textbf{Error tracing}: Follow exception paths through multiple modules and call stacks
    \item \textbf{Race condition identification}: Debug concurrent/async code issues
    \item \textbf{Performance debugging}: Identify bottlenecks across multiple components
    \item \textbf{Integration failures}: Debug API communication and data flow issues
    \item \textbf{Context}: 30K-120K tokens, real production bug scenarios
\end{itemize}

\subsubsection{5. Multi-session Development (1,200 instances)}
Novel evaluation of context persistence across development sessions:
\begin{itemize}
    \item \textbf{Session continuity}: Maintain architectural decisions across 3-5 sessions
    \item \textbf{Incremental feature building}: Build complex features across multiple sessions
    \item \textbf{Context-aware modifications}: Respect previous decisions when adding functionality
    \item \textbf{State management}: Track changes and dependencies across sessions
    \item \textbf{Context}: 40K-200K tokens, extended development workflows
\end{itemize}

\subsubsection{6. Code Comprehension (1,500 instances)}
Tests deep understanding and explanation capabilities:
\begin{itemize}
    \item \textbf{Algorithm explanation}: Describe complex algorithms and their purpose
    \item \textbf{Architecture documentation}: Generate comprehensive system documentation
    \item \textbf{Code review}: Identify issues, improvements, and architectural concerns
    \item \textbf{Knowledge extraction}: Extract patterns, practices, and design principles
    \item \textbf{Context}: 10K-80K tokens, focus on understanding over generation
\end{itemize}

\subsubsection{7. Integration Testing (1,300 instances)}
Evaluates understanding of system integration and testing:
\begin{itemize}
    \item \textbf{Component interaction testing}: Verify interfaces between modules
    \item \textbf{API compatibility validation}: Ensure backward compatibility in changes
    \item \textbf{End-to-end workflow testing}: Test complete user journeys across systems
    \item \textbf{Deployment validation}: Verify changes work in production-like environments
    \item \textbf{Context}: 35K-100K tokens, system-level testing scenarios
\end{itemize}

\subsubsection{8. Security Analysis (1,200 instances)}
Tests security-aware development practices:
\begin{itemize}
    \item \textbf{Vulnerability detection}: Identify SQL injection, XSS, authentication flaws
    \item \textbf{Access control analysis}: Verify proper authorization across endpoints
    \item \textbf{Data flow security}: Trace sensitive data through system components
    \item \textbf{Secure coding practices}: Implement security patterns and best practices
    \item \textbf{Context}: 20K-80K tokens, security-critical code scenarios
\end{itemize}

\subsection{Difficulty Levels and Context Distribution}

AgentCodeEval implements a sophisticated difficulty progression based on context length, architectural complexity, and information coverage:

\begin{enumerate}
    \item \textbf{Easy Level (3,200 instances)}
    \begin{itemize}
        \item \textbf{Context Length}: 10K-40K tokens (15-30 files)
        \item \textbf{Complexity}: Basic architectural patterns, straightforward dependencies
        \item \textbf{Tasks}: Simple refactoring, basic feature addition, clear bug scenarios
        \item \textbf{Information Coverage}: IC = 0.7-0.8 (moderate context utilization)
        \item \textbf{Session Count}: 1-2 sessions for multi-session tasks
    \end{itemize}
    
    \item \textbf{Medium Level (4,500 instances)}
    \begin{itemize}
        \item \textbf{Context Length}: 40K-100K tokens (30-80 files)
        \item \textbf{Complexity}: Multiple architectural layers, moderate dependency chains
        \item \textbf{Tasks}: Cross-module refactoring, complex feature integration, multi-component debugging
        \item \textbf{Information Coverage}: IC = 0.8-0.9 (high context utilization)
        \item \textbf{Session Count}: 2-4 sessions for multi-session tasks
    \end{itemize}
    
    \item \textbf{Hard Level (3,600 instances)}
    \begin{itemize}
        \item \textbf{Context Length}: 100K-200K tokens (80-200+ files)
        \item \textbf{Complexity}: Complex microservice architectures, deep dependency networks
        \item \textbf{Tasks}: System-wide changes, architectural refactoring, complex integration scenarios
        \item \textbf{Information Coverage}: IC > 0.9 (very high context utilization)
        \item \textbf{Session Count}: 3-5 sessions for multi-session tasks
    \end{itemize}
    
    \item \textbf{Expert Level (700 instances)}
    \begin{itemize}
        \item \textbf{Context Length}: 200K+ tokens (200+ files, full repository context)
        \item \textbf{Complexity}: Enterprise-scale architectures, maximum complexity
        \item \textbf{Tasks}: Full-system understanding, major architectural decisions
        \item \textbf{Information Coverage}: IC > 0.95 (near-complete context utilization)
        \item \textbf{Session Count}: 5+ sessions for extended development workflows
    \end{itemize}
\end{enumerate}

\subsection{Progressive Complexity Design}
Each difficulty level introduces additional challenges:
\begin{itemize}
    \item \textbf{Architectural Depth}: From single-layer to multi-tier architectures
    \item \textbf{Dependency Complexity}: From linear to circular and nested dependencies
    \item \textbf{Context Span}: From local modules to entire repository understanding
    \item \textbf{Session Memory**: From short-term to long-term context retention
\end{itemize}

\section{Data Sources and Infrastructure}

\subsection{Foundation Datasets}
\begin{table}[h]
\centering
\begin{tabular}{@{}lll@{}}
\toprule
Dataset & Size & Description \\
\midrule
The Stack v2 (BigCode) & 67.5TB & 3B+ files, 600+ languages, production codebases \\
StarCoderData & 783GB & Refined, deduplicated version with PII removal \\
Project CodeNet (IBM) & 14M samples & Problem-solution pairs across 4K problems \\
CodeSearchNet & 2M pairs & Comment-code pairs for documentation tasks \\
\bottomrule
\end{tabular}
\caption{Primary data sources for AgentCodeEval generation}
\end{table}

\subsection{Code Analysis Infrastructure}
\begin{itemize}
    \item \textbf{Tree-sitter}: Multi-language incremental parser supporting 358+ languages
    \item \textbf{rust-code-analysis (Mozilla)}: Production-grade metrics extraction and complexity analysis
    \item \textbf{metric-gardener}: Multi-language static analysis for dependency mapping
    \item \textbf{tree-climber}: Program analysis tools for control flow and data flow analysis
\end{itemize}

\subsection{Evaluation Framework Integration}
\begin{itemize}
    \item \textbf{LightEval (Hugging Face)}: Extended for agent-specific evaluation with vLLM support
    \item \textbf{LOOM-Scope}: Long-context evaluation framework integration
    \item \textbf{Custom Agent Metrics}: Novel metrics for architectural coherence and memory retention
    \item \textbf{LLM-as-a-Judge}: Automated evaluation using GPT-4o/Claude for complex reasoning assessment
\end{itemize}

\subsection{Repository Selection Criteria}
From The Stack v2's 3B+ files, repositories are selected based on:
\begin{itemize}
    \item \textbf{Quality Metrics}: >100 stars, active development, clean commit history
    \item \textbf{Architectural Complexity}: Multiple modules, clear patterns, realistic dependencies
    \item \textbf{Size Range}: 50-500 files per repository (10K-200K tokens)
    \item \textbf{Language Diversity}: Python, JavaScript, Java, C++, Go, TypeScript
    \item \textbf{Domain Coverage}: Web frameworks, ML libraries, system tools, enterprise applications
\end{itemize}

\subsection{Target Repository Categories}
\begin{table}[h]
\centering
\begin{tabular}{@{}lll@{}}
\toprule
Category & Examples & Instance Count \\
\midrule
ML/AI Frameworks & TensorFlow, PyTorch, scikit-learn & 2,400 \\
Web Frameworks & React, Vue, Django, Flask & 2,400 \\
System Tools & Docker, Kubernetes, Git & 1,800 \\
Enterprise Apps & ERP systems, CMS platforms & 1,800 \\
API Services & REST APIs, GraphQL services & 1,800 \\
Database Systems & ORMs, migration tools & 1,200 \\
Security Tools & Authentication, encryption & 1,200 \\
DevOps Tools & CI/CD, monitoring & 600 \\
\bottomrule
\end{tabular}
\caption{Repository distribution for comprehensive domain coverage}
\end{table}

\section{Methodology}

\subsection{Automated Generation Pipeline}

AgentCodeEval employs a sophisticated 4-phase automated generation pipeline to create 12,000 high-quality evaluation instances:

\subsubsection{Phase 1: Repository Analysis and Selection}
\begin{enumerate}
    \item \textbf{Repository Filtering}: Process The Stack v2's 3B+ files using quality metrics (stars >100, clean history)
    \item \textbf{AST Parsing}: Use tree-sitter to parse selected repositories into language-agnostic ASTs
    \item \textbf{Dependency Analysis}: Extract import graphs, function call chains, and module relationships
    \item \textbf{Architectural Pattern Detection}: Identify MVC, microservice, layered architectures using pattern matching
    \item \textbf{Complexity Metrics}: Calculate cyclomatic complexity, dependency depth, and cohesion metrics
    \item \textbf{Output}: 50,000 analyzed repositories with rich metadata
\end{enumerate}

\subsubsection{Phase 2: Scenario Template Generation}
\begin{enumerate}
    \item \textbf{Template Creation}: Develop 15 scenario templates per task category (120 total)
    \item \textbf{Context Window Planning}: Design progressive complexity for 10K-200K token ranges
    \item \textbf{Information Coverage Optimization**: Ensure IC > 0.7 by distributing critical information across context
    \item \textbf{Multi-session Workflow Design**: Create 2-5 session templates for development workflows
    \item \textbf{Validation Framework**: Implement automated checks for task solvability and complexity
    \item \textbf{Output**: 120 validated scenario templates with complexity calibration
\end{enumerate}

\subsubsection{Phase 3: Instance Generation at Scale}
\begin{enumerate}
    \item \textbf{Template-Repository Matching**: Pair scenarios with appropriate repositories based on complexity
    \item \textbf{Synthetic Variation Generation**: Create 100 variations per template using LLM APIs
    \item \textbf{Context Length Distribution**: Ensure proper distribution across Easy/Medium/Hard/Expert levels
    \item \textbf{Quality Assurance Pipeline**: Automated validation of syntax, dependencies, and information coverage
    \item \textbf{Deduplication**: Remove near-duplicate instances using semantic similarity
    \item \textbf{Output**: 12,000 unique, validated evaluation instances
\end{enumerate}

\subsubsection{Phase 4: Reference Solution and Evaluation Creation}
\begin{enumerate}
    \item \textbf{Multi-LLM Solution Generation**: Use GPT-4o, Claude 3.5, and Gemini for diverse reference solutions
    \item \textbf{Solution Validation Framework**: Automated testing of generated solutions for correctness
    \item \textbf{Multiple Solution Variants**: 2-3 reference solutions per instance for comprehensive evaluation
    \item \textbf{Evaluation Rubric Creation**: Detailed scoring criteria for each task category
    \item \textbf{Human Quality Sampling**: Manual validation of 5% (600 instances) across all categories
    \item \textbf{Output**: Validated benchmark with reference solutions and evaluation framework
\end{enumerate}

\subsection{Novel Agent-Specific Evaluation Metrics}

AgentCodeEval introduces comprehensive evaluation metrics specifically designed for software development agents:

\subsubsection{Architectural Coherence Score (ACS)}
Measures consistency with existing architectural patterns and design principles:
\begin{itemize}
    \item \textbf{Pattern Adherence}: Consistency with MVC, Observer, Factory patterns (0-1 scale)
    \item \textbf{Layer Separation**: Proper separation between presentation, business, data layers
    \item \textbf{Module Cohesion**: Logical grouping of related functionality
    \item \textbf{Interface Consistency**: Adherence to established API contracts
    \item \textbf{Formula}: ACS = 0.3×Pattern + 0.3×Separation + 0.2×Cohesion + 0.2×Interface
\end{itemize}

\subsubsection{Dependency Traversal Accuracy (DTA)}
Evaluates correct navigation and understanding of complex dependency networks:
\begin{itemize}
    \item \textbf{Import Resolution**: Correct identification of required dependencies
    \item \textbf{Call Chain Accuracy**: Proper understanding of function call sequences
    \item \textbf{Circular Dependency Detection**: Identification and handling of dependency cycles
    \item \textbf{Abstraction Level Respect**: Maintaining proper architectural boundaries
    \item \textbf{Formula}: DTA = 0.4×Import + 0.3×CallChain + 0.15×Circular + 0.15×Abstraction
\end{itemize}

\subsubsection{Multi-Session Memory Retention (MMR)}
Novel metric for evaluating context persistence across development sessions:
\begin{itemize}
    \item \textbf{Decision Consistency**: Maintaining architectural decisions across sessions
    \item \textbf{Context Carryover**: Remembering constraints and requirements from previous sessions
    \item \textbf{Incremental Building**: Successfully building on previous implementations
    \item \textbf{State Tracking**: Maintaining awareness of changes and their implications
    \item \textbf{Formula**: MMR = 0.3×Decision + 0.3×Context + 0.25×Building + 0.15×State
\end{itemize}

\subsubsection{Cross-File Reasoning Depth (CFRD)}
Measures understanding of relationships across multiple files:
\begin{itemize}
    \item \textbf{Inter-module Understanding**: Comprehension of cross-file relationships
    \item \textbf{Data Flow Tracking**: Following data transformations across modules
    \item \textbf{Control Flow Analysis**: Understanding execution paths across files
    \item \textbf{Side Effect Recognition**: Identifying cross-module side effects
    \item \textbf{Formula**: CFRD = 0.3×Inter + 0.25×DataFlow + 0.25×ControlFlow + 0.2×SideEffect
\end{itemize}

\subsubsection{Incremental Development Capability (IDC)}
Evaluates ability to build upon existing implementations progressively:
\begin{itemize}
    \item \textbf{Feature Extension**: Successfully adding functionality to existing systems
    \item \textbf{Refactoring Preservation**: Maintaining functionality during restructuring
    \item \textbf{Integration Compatibility**: Ensuring new features work with existing components
    \item \textbf{Regression Prevention**: Avoiding introduction of bugs in existing functionality
    \item \textbf{Formula**: IDC = 0.3×Extension + 0.25×Preservation + 0.25×Integration + 0.2×Prevention
\end{itemize}

\subsubsection{Information Coverage Utilization (ICU)}
Measures how effectively agents use the provided context:
\begin{itemize}
    \item \textbf{Context Span**: Percentage of provided context actually referenced
    \item \textbf{Critical Information Identification**: Recognition of key architectural components
    \item \textbf{Relevance Filtering**: Ability to distinguish relevant from irrelevant information
    \item \textbf{Deep vs Surface Understanding**: Quality of context utilization
    \item \textbf{Formula**: ICU = 0.3×Span + 0.3×Critical + 0.2×Filtering + 0.2×Depth
\end{itemize}

\subsubsection{Composite Agent Development Score (CADS)}
Overall agent capability combining all metrics:
\begin{itemize}
    \item \textbf{CADS = 0.2×ACS + 0.2×DTA + 0.2×MMR + 0.15×CFRD + 0.15×IDC + 0.1×ICU}
    \item \textbf{Range**: 0-5 scale for consistency with existing benchmarks
    \item \textbf{Interpretation**: >4.0 (Excellent), 3.0-4.0 (Good), 2.0-3.0 (Fair), <2.0 (Poor)
\end{itemize}

\section{Achieving 12,000 Instance Scale}

\subsection{Instance Distribution Strategy}
AgentCodeEval's 12,000 instances are strategically distributed to ensure comprehensive coverage:

\begin{table}[h]
\centering
\begin{tabular}{@{}lrrrr@{}}
\toprule
Task Category & Easy & Medium & Hard & Total \\
\midrule
Architectural Understanding & 400 & 600 & 500 & 1,500 \\
Cross-file Refactoring & 400 & 600 & 500 & 1,500 \\
Feature Implementation & 500 & 700 & 600 & 1,800 \\
Bug Investigation & 400 & 600 & 500 & 1,500 \\
Multi-session Development & 300 & 500 & 400 & 1,200 \\
Code Comprehension & 500 & 600 & 400 & 1,500 \\
Integration Testing & 400 & 500 & 400 & 1,300 \\
Security Analysis & 400 & 500 & 300 & 1,200 \\
\midrule
\textbf{Total} & \textbf{3,200} & \textbf{4,500} & \textbf{3,600} & \textbf{12,000} \\
\bottomrule
\end{tabular}
\caption{AgentCodeEval instance distribution across categories and difficulty levels}
\end{table}

\subsection{Context Length Distribution}
\begin{itemize}
    \item \textbf{Short Context (10K-40K tokens)}: 3,200 instances (Easy tasks)
    \item \textbf{Medium Context (40K-100K tokens)}: 4,500 instances (Medium tasks)
    \item \textbf{Long Context (100K-200K tokens)}: 3,600 instances (Hard tasks)
    \item \textbf{Ultra-Long Context (200K+ tokens)}: 700 instances (Expert-level evaluation)
\end{itemize}

\subsection{Scalability Advantages}

\subsubsection{Automated Generation Pipeline}
Our scale is achievable through automated generation using The Stack v2:
\begin{itemize}
    \item \textbf{Repository Pool}: 50,000+ high-quality repositories
    \item \textbf{Synthetic Scenario Generation}: LLM-generated realistic development scenarios
    \item \textbf{Template-based Scaling}: 8 task categories × 15 scenario templates = 120 base patterns
    \item \textbf{Variation Generation}: 100 variations per base pattern = 12,000 instances
\end{itemize}

\subsubsection{Quality Control at Scale}
\begin{itemize}
    \item \textbf{Automated Validation}: Tree-sitter parsing ensures syntactic correctness
    \item \textbf{Complexity Verification}: Information Coverage (IC) > 0.7 for all instances
    \item \textbf{Human Sampling}: Manual validation of 5\% (600 instances) across all categories
    \item \textbf{Difficulty Calibration}: Progressive complexity scaling within categories
\end{itemize}

\subsubsection{Resource Efficiency}
Despite 12,000 instances, resource requirements remain manageable:
\begin{itemize}
    \item \textbf{Generation Cost}: \$3,000 in API calls (Claude/GPT-4 for reference solutions)
    \item \textbf{Storage Requirements}: 50GB compressed benchmark data
    \item \textbf{Evaluation Time}: 200 hours distributed across multiple API keys
    \item \textbf{No Human Annotation}: Fully automated generation and validation
\end{itemize}

\section{Implementation Plan}

\subsection{Phase 1: Infrastructure and Foundation (Weeks 1-2)}
\begin{itemize}
    \item \textbf{Data Access Setup}: Configure API access to The Stack v2 and StarCoderData
    \item \textbf{Analysis Pipeline**: Deploy tree-sitter, rust-code-analysis, and metric-gardener
    \item \textbf{Repository Selection**: Filter and select 50,000 high-quality repositories from The Stack v2
    \item \textbf{AST Processing Pipeline**: Implement automated parsing for Python, JavaScript, Java, C++, Go
    \item \textbf{Quality Metrics Framework**: Establish repository quality scoring (stars, complexity, patterns)
    \item \textbf{Deliverable**: 50,000 analyzed repositories with rich architectural metadata
\end{itemize}

\subsection{Phase 2: Template and Scenario Design (Weeks 3-4)}
\begin{itemize}
    \item \textbf{Task Category Templates**: Create 15 scenario templates per category (120 total)
    \item \textbf{Complexity Calibration**: Design progressive difficulty scaling for context lengths
    \item \textbf{Information Coverage Optimization**: Implement ETHIC-style IC > 0.7 requirement
    \item \textbf{Multi-session Framework**: Design 2-5 session development workflow templates
    \item \textbf{Validation Pipeline**: Automated solvability and complexity verification
    \item \textbf{Deliverable**: 120 validated scenario templates with complexity calibration
\end{itemize}

\subsection{Phase 3: Large-Scale Instance Generation (Weeks 5-6)}
\begin{itemize}
    \item \textbf{Template-Repository Matching**: Intelligent pairing based on complexity and domain
    \item \textbf{Automated Variation Generation**: LLM APIs (GPT-4o, Claude 3.5) for 100 variations per template
    \item \textbf{Quality Assurance Pipeline**: Syntax validation, dependency checking, IC verification
    \item \textbf{Deduplication System**: Semantic similarity-based duplicate removal
    \item \textbf{Distribution Validation**: Ensure proper Easy/Medium/Hard/Expert distribution
    \item \textbf{Deliverable**: 12,000 unique, validated evaluation instances
\end{itemize}

\subsection{Phase 4: Evaluation Framework and Model Assessment (Weeks 7-8)}
\begin{itemize}
    \item \textbf{Agent-Specific Metrics**: Implement ACS, DTA, MMR, CFRD, IDC, ICU metrics
    \item \textbf{Reference Solution Generation**: Multi-LLM approach for 2-3 solutions per instance
    \item \textbf{Evaluation Infrastructure**: Extend LightEval for agent-specific evaluation
    \item \textbf{Model Evaluation Campaign**: Assess 16-20 models (commercial + open-source)
    \item \textbf{Statistical Analysis Framework**: Performance analysis across multiple dimensions
    \item \textbf{Deliverable**: Complete benchmark with evaluation results for 16-20 models
\end{itemize}

\subsection{Phase 5: Validation, Documentation and Release (Weeks 9-10)}
\begin{itemize}
    \item \textbf{Human Validation Study**: Manual validation of 5% (600 instances) across categories
    \item \textbf{Benchmark Reliability Analysis**: Inter-rater agreement, statistical significance testing
    \item \textbf{Comprehensive Documentation**: Technical documentation, usage guides, evaluation protocols
    \item \textbf{Research Paper Preparation**: Write and submit to top-tier venue (ICLR/NeurIPS/ICML)
    \item \textbf{Public Release**: Open-source benchmark, leaderboard, and evaluation tools
    \item \textbf{Deliverable**: Published benchmark, research paper, and public evaluation platform
\end{itemize}

\subsection{Success Metrics and Milestones}
\begin{itemize}
    \item \textbf{Scale Achievement**: 12,000 instances (5x larger than SWE-Bench)
    \item \textbf{Quality Validation**: IC > 0.7 for all instances, 95\% human validation agreement
    \item \textbf{Model Discrimination**: Clear performance differences across difficulty levels
    \item \textbf{Statistical Significance**: Robust performance differences between model categories
    \item \textbf{Community Adoption**: >50 model submissions within 6 months of release
\end{itemize}

\section{Expected Contributions and Impact}

\subsection{Primary Research Contributions}
\begin{enumerate}
    \item \textbf{Unprecedented Scale Benchmark}: First 12,000-instance benchmark for software development agents—5x larger than existing code benchmarks
    \item \textbf{Novel Agent-Specific Evaluation}: Introduction of 6 new metrics (ACS, DTA, MMR, CFRD, IDC, ICU) specifically designed for development agents
    \item \textbf{Multi-Session Evaluation Paradigm}: First benchmark to evaluate context persistence and incremental development across multiple sessions
    \item \textbf{Long-Context Code Integration**: Comprehensive evaluation of 10K-200K token codebases with high information coverage (IC > 0.7)
    \item \textbf{Automated Generation Pipeline**: Scalable methodology for creating realistic development scenarios using production codebases
    \item \textbf{Architectural Understanding Assessment}: First systematic evaluation of design pattern recognition and architectural comprehension at scale
\end{enumerate}

\subsection{Technical Innovations}
\begin{enumerate}
    \item \textbf{Information Coverage for Code**: Extension of ETHIC methodology to software development with distributed dependencies
    \item \textbf{Production Codebase Integration**: Utilization of The Stack v2's 3B+ files for authentic development scenarios
    \item \textbf{Multi-LLM Reference Generation**: Robust evaluation using diverse reference solutions from multiple state-of-the-art models
    \item \textbf{Progressive Complexity Scaling**: Sophisticated difficulty progression from 10K to 200K+ tokens with calibrated complexity
    \item \textbf{Agent Workflow Simulation**: Realistic multi-step development processes spanning architecture, implementation, debugging, and maintenance
\end{enumerate}

\subsection{Practical Impact}
\begin{enumerate}
    \item \textbf{Industry Standard Establishment**: Potential to become the definitive benchmark for software development agents (like HumanEval for code completion)
    \item \textbf{Model Development Guidance**: Clear metrics for improving long-context architectural understanding and memory retention
    \item \textbf{Research Acceleration**: Large-scale dataset enabling statistical analysis of agent capabilities across multiple dimensions
    \item \textbf{Commercial Development Support**: Rigorous evaluation framework for AI-powered development tools and coding assistants
    \item \textbf{Educational Applications**: Training and assessment framework for software engineering education
    \item \textbf{Open Science Advancement**: Fully open-source benchmark, evaluation tools, and comprehensive documentation
\end{enumerate}

\subsection{Long-term Research Implications}
\begin{enumerate}
    \item \textbf{Long-Context Model Development**: Drive improvements in architectural understanding and memory retention for long-context LLMs
    \item \textbf{Agent Architecture Research}: Enable systematic study of multi-session development workflows and context persistence
    \item \textbf{Software Engineering AI**: Establish foundation for AI-assisted software development research and practice
    \item \textbf{Benchmark Methodology**: Demonstrate scalable approaches for creating large-scale, high-quality evaluation datasets
    \item \textbf{Interdisciplinary Impact**: Bridge software engineering and AI research communities through shared evaluation standards
\end{enumerate}

\subsection{Expected Community Adoption}
\begin{itemize}
    \item \textbf{Academic Research**: >100 research papers citing AgentCodeEval within 2 years
    \item \textbf{Industry Adoption**: Integration into major AI development platforms and evaluation suites
    \item \textbf{Model Benchmarking**: Standard evaluation for all major LLM releases in software development domain
    \item \textbf{Educational Integration**: Adoption in computer science curricula for AI and software engineering courses
    \item \textbf{Open Source Development**: Active community contributions and extensions to the benchmark framework
\end{itemize}

\section{Resource Requirements}

\subsection{Computational Resources}
\begin{itemize}
    \item Standard server with 64GB+ RAM (no GPU required for analysis)
    \item 500GB+ storage for datasets and 12,000 generated tasks
    \item API access to Claude/GPT-4/Gemini for reference generation
    \item Distributed API key management for parallel generation (estimated \$3,000 total cost)
\end{itemize}

\subsection{No Requirements}
\begin{itemize}
    \item Human annotation or labeling
    \item GPU resources for training
    \item Complex distributed infrastructure
\end{itemize}

\section{Timeline and Deliverables}

\subsection{10-Week Timeline}
\begin{itemize}
    \item Weeks 1-2: Foundation setup and tool configuration
    \item Weeks 3-4: Task generation pipeline development
    \item Weeks 5-6: Reference solution creation and validation
    \item Weeks 7-8: Benchmark assembly and initial evaluation
    \item Weeks 9-10: Validation, documentation, and publication
\end{itemize}

\subsection{Benchmark Scale Comparison}
\begin{table}[h]
\centering
\begin{tabular}{@{}lrr@{}}
\toprule
Benchmark & Instances & Context Length \\
\midrule
HumanEval & 164 & Single function \\
MBPP & 1,000 & Single function \\
SWE-Bench & 2,294 & Repository-level \\
L-Eval & 2,000+ & 3K-200K tokens \\
ETHIC & 1,986 & High coverage \\
AgentRewardBench & 1,302 & Web trajectories \\
\textbf{AgentCodeEval} & \textbf{12,000} & \textbf{10K-200K tokens} \\
\bottomrule
\end{tabular}
\caption{Benchmark scale comparison - AgentCodeEval provides 5x more instances than the largest existing benchmark}
\end{table}

\subsection{Key Deliverables}
\begin{enumerate}
    \item AgentCodeEval benchmark dataset (12,000 instances)
    \item Automated evaluation framework and metrics
    \item Baseline performance analysis of major LLMs
    \item Research paper documenting methodology and results
    \item Open-source release of tools and benchmark
\end{enumerate}

\section{Conclusion}

AgentCodeEval addresses a critical gap in long-context LLM evaluation by focusing on realistic software development agent tasks. With 12,000 instances—5x larger than the current largest benchmark—AgentCodeEval provides unprecedented comprehensive coverage of agent-based software development scenarios. By leveraging existing high-quality code datasets and automated analysis tools, we can create this large-scale benchmark without requiring extensive human annotation or GPU resources.

The proposed approach is both novel and practical, offering a scalable methodology for generating realistic evaluation scenarios while maintaining rigorous evaluation standards. The 12,000-instance scale enables statistically significant analysis across multiple dimensions: context length, task complexity, programming languages, and architectural patterns. AgentCodeEval has the potential to become the definitive standard benchmark in the field, establishing new baselines for long-context software development agent evaluation.

\section{AgentCodeEval vs. Existing Benchmarks: Key Differentiators}

\subsection{Unique Positioning}
AgentCodeEval distinguishes itself from existing evaluation frameworks through several key innovations:

\begin{table}[h]
\centering
\begin{tabular}{@{}lllll@{}}
\toprule
Benchmark & Context Length & Task Type & Agent Focus & Evaluation Scope \\
\midrule
HumanEval & Single function & Code completion & None & Function-level \\
MBPP & Single function & Code generation & None & Problem-solving \\
SWE-Bench & Repository & Single issue & Limited & Issue resolution \\
L-Eval & 3K-200K tokens & General long-context & None & Document QA \\
∞Bench & 100K+ tokens & General reasoning & None & Long context \\
ETHIC & High coverage & Domain tasks & None & Information utilization \\
AgentRewardBench & Web trajectories & Web agents & Web-specific & Trajectory evaluation \\
\textbf{AgentCodeEval} & \textbf{10K-200K tokens} & \textbf{Multi-file development} & \textbf{Code agents} & \textbf{Agent workflows} \\
\bottomrule
\end{tabular}
\caption{Comparison of AgentCodeEval with existing benchmarks}
\end{table}

\subsection{Novel Evaluation Dimensions}

\subsubsection{Multi-Session Development Workflows}
Unlike existing benchmarks that evaluate single-turn interactions, AgentCodeEval introduces \textbf{multi-session evaluation}:
\begin{itemize}
    \item Session 1: Initial implementation and architecture setup
    \item Session 2: Feature addition and integration
    \item Session 3: Bug fixing and optimization
    \item Session 4: Documentation and testing
    \item Session 5: Refactoring and maintenance
\end{itemize}

\subsubsection{Architectural Understanding at Scale}
AgentCodeEval is the first benchmark to systematically evaluate \textbf{architectural comprehension}:
\begin{itemize}
    \item Design pattern recognition across 50+ files
    \item Dependency graph understanding with 100+ components
    \item Microservice architecture navigation
    \item Database schema integration reasoning
\end{itemize}

\subsubsection{Long-Context Agent Memory}
Novel evaluation of \textbf{context persistence} and \textbf{incremental development}:
\begin{itemize}
    \item Maintaining architectural decisions across sessions
    \item Building upon previous implementation choices
    \item Context-aware code generation that respects existing patterns
    \item Memory of cross-file dependencies and constraints
\end{itemize}

\subsection{Technical Innovation}

\subsubsection{High Information Coverage for Code}
Extending ETHIC's Information Coverage (IC) methodology to software development:
\begin{itemize}
    \item Code IC > 0.7: Tasks require 70%+ of provided codebase
    \item Distributed dependencies: No single file contains the solution
    \item Cross-module reasoning: Solutions span multiple architectural layers
\end{itemize}

\subsubsection{Agent-Specific Evaluation Metrics}
Novel metrics designed specifically for code agent evaluation:
\begin{itemize}
    \item \textbf{Architectural Coherence Score (ACS)}: Measures consistency with existing patterns
    \item \textbf{Dependency Traversal Accuracy (DTA)}: Correct navigation of complex dependencies
    \item \textbf{Multi-Session Memory Retention (MMR)}: Context maintenance across sessions
    \item \textbf{Cross-File Reasoning Depth (CFRD)}: Understanding multi-file relationships
    \item \textbf{Incremental Development Capability (IDC)}: Building on previous work
\end{itemize}

\subsubsection{Real-World Codebase Complexity}
AgentCodeEval uses actual production codebases from The Stack v2:
\begin{itemize}
    \item Real architectural patterns from 1000+ repositories
    \item Authentic dependency structures and complexity
    \item Production-level code quality and conventions
    \item Diverse programming paradigms and languages
\end{itemize}

\end{document}
