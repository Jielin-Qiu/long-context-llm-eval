\documentclass{article}
\usepackage[utf8]{inputenc}
\usepackage{amsmath}
\usepackage{amsfonts}
\usepackage{amssymb}
\usepackage{graphicx}
\usepackage{hyperref}
\usepackage{enumitem}
\usepackage{booktabs}
\usepackage{geometry}
\geometry{margin=1in}

\title{AgentCodeEval: A Novel Benchmark for Evaluating Long-Context Language Models in Software Development Agent Tasks}
\author{Research Proposal}
\date{\today}

\begin{document}

\maketitle

\begin{abstract}
We propose AgentCodeEval, a comprehensive benchmark for evaluating long-context language models (LLMs) in software development agent scenarios. Unlike existing code evaluation benchmarks that focus on single-function completion or isolated code understanding, AgentCodeEval addresses the critical gap in evaluating LLMs' ability to perform complex, multi-file software development tasks that require understanding large codebases, managing dependencies, and implementing features across multiple sessions. This benchmark targets real-world software development workflows where agents must navigate complex file structures, understand inter-module dependencies, and maintain architectural consistency across large-scale codebases. With 8,000 evaluation instances, AgentCodeEval provides 3.5x more comprehensive coverage than the largest existing benchmark (SWE-Bench: 2,294 instances).
\end{abstract}

\section{Introduction}

\subsection{Motivation}
Current long-context LLM evaluation benchmarks exhibit significant limitations when assessing software development agent capabilities:

\subsubsection{Existing Benchmark Limitations}
\begin{itemize}
    \item \textbf{HumanEval (164 instances)}: Single-function code completion, no architectural context
    \item \textbf{MBPP (1,000 instances)}: Isolated problem-solving, no codebase integration
    \item \textbf{SWE-Bench (2,294 instances)}: Single-issue resolution, limited agent workflow evaluation
    \item \textbf{L-Eval (2,000+ instances)}: General long-context tasks, no code-specific evaluation
    \item \textbf{ETHIC (1,986 instances)}: High information coverage but not code-focused
\end{itemize}

\subsubsection{Real-World Agent Requirements}
Software development agents must handle complex, interconnected tasks that existing benchmarks cannot evaluate:
\begin{itemize}
    \item \textbf{Multi-file codebase understanding}: Navigate 50+ files with complex dependencies
    \item \textbf{Cross-module dependency management}: Understand and maintain architectural relationships
    \item \textbf{Multi-session development workflows}: Maintain context across 2-5 development sessions
    \item \textbf{Architectural pattern recognition}: Identify and maintain design patterns at scale
    \item \textbf{Integration of new features}: Add functionality while preserving existing architecture
    \item \textbf{Long-context memory retention}: Remember decisions and constraints across 200K+ tokens
\end{itemize}

\subsection{Research Gap}
Despite the critical importance of AI-powered development tools, no existing benchmark comprehensively evaluates long-context LLMs in realistic software development agent scenarios. Current evaluation methods fall short in three key areas:

\begin{enumerate}
    \item \textbf{Scale}: Largest code benchmark (SWE-Bench) has only 2,294 instances
    \item \textbf{Agent Focus}: No benchmark specifically designed for agent workflows
    \item \textbf{Long-Context Integration}: No evaluation of code understanding at 100K+ token scale
\end{enumerate}

\section{AgentCodeEval: Benchmark Design}

\subsection{Core Concept}
AgentCodeEval evaluates LLMs on realistic software development scenarios that span multiple files, require understanding of complex dependencies, and simulate multi-session development workflows. With 8,000 instances across 8 comprehensive task categories, AgentCodeEval provides 3.5x more extensive coverage than existing benchmarks.

\subsection{Key Design Principles}
\begin{enumerate}
    \item \textbf{Unprecedented Scale}: 8,000 instances—3.5x larger than SWE-Bench (2,294)
    \item \textbf{Real-world Relevance}: Tasks derived from actual production codebases in The Stack v2
    \item \textbf{Multi-file Complexity}: Each task involves 15-200+ files with authentic dependencies
    \item \textbf{Long-context Requirement}: Tasks require understanding of 10K-200K+ tokens
    \item \textbf{Agent-specific Evaluation}: Focus on multi-step agent workflows, not single-turn completion
    \item \textbf{High Information Coverage}: Extending ETHIC methodology with IC > 0.7 for all tasks
    \item \textbf{Automated Generation Pipeline}: Scalable creation using tree-sitter and LLM APIs
    \item \textbf{Novel Evaluation Metrics}: Agent-specific metrics for architectural understanding and memory retention
\end{enumerate}

\subsection{Benchmark Architecture}
AgentCodeEval consists of three integrated evaluation layers:
\begin{itemize}
    \item \textbf{Task Layer}: 8 categories covering all aspects of software development agents
    \item \textbf{Context Layer}: Progressive complexity from 10K to 200K+ tokens
    \item \textbf{Evaluation Layer}: Multi-dimensional metrics including architectural coherence and memory retention
\end{itemize}

\subsection{Task Categories}

AgentCodeEval comprises 8 comprehensive task categories, each designed to evaluate specific agent capabilities:

\subsubsection{1. Architectural Understanding (1,000 instances)}
Evaluates agents' ability to comprehend and work with large-scale software architecture:
\begin{itemize}
    \item \textbf{Design pattern recognition}: Identify MVC, Observer, Factory patterns across 50+ files
    \item \textbf{Dependency graph analysis}: Understand component relationships in microservice architectures
    \item \textbf{Layered architecture navigation}: Work with presentation, business, and data layers
    \item \textbf{Context}: 15K-80K tokens, real production codebases from The Stack v2
\end{itemize}

\subsubsection{2. Cross-file Refactoring (1,000 instances)}
Tests systematic code restructuring while maintaining functionality:
\begin{itemize}
    \item \textbf{Extract functionality}: Move shared code into new modules across 20+ files
    \item \textbf{Rename operations}: Update class/function names consistently across codebase
    \item \textbf{Package restructuring}: Reorganize module hierarchies while preserving imports
    \item \textbf{Design pattern application}: Implement patterns like Strategy or Command across multiple files
    \item \textbf{Context}: 20K-100K tokens, complex dependency chains
\end{itemize}

\subsubsection{3. Feature Implementation (1,000 instances)}
Evaluates adding new functionality to existing systems:
\begin{itemize}
    \item \textbf{Authentication systems}: Implement OAuth/JWT across web application layers
    \item \textbf{API endpoint creation}: Add REST endpoints with validation, routing, and documentation
    \item \textbf{Database integration}: Implement new schemas while maintaining data consistency
    \item \textbf{Third-party integration}: Add external services (payment, notification, analytics)
    \item \textbf{Context}: 25K-150K tokens, full-stack development scenarios
\end{itemize}

\subsubsection{4. Bug Investigation (1,000 instances)}
Tests systematic debugging across complex codebases:
\begin{itemize}
    \item \textbf{Error tracing}: Follow exception paths through multiple modules and call stacks
    \item \textbf{Race condition identification}: Debug concurrent/async code issues
    \item \textbf{Performance debugging}: Identify bottlenecks across multiple components
    \item \textbf{Integration failures}: Debug API communication and data flow issues
    \item \textbf{Context}: 30K-120K tokens, real production bug scenarios
\end{itemize}

\subsubsection{5. Multi-session Development (1,000 instances)}
Novel evaluation of context persistence across development sessions:
\begin{itemize}
    \item \textbf{Session continuity}: Maintain architectural decisions across 3-5 sessions
    \item \textbf{Incremental feature building}: Build complex features across multiple sessions
    \item \textbf{Context-aware modifications}: Respect previous decisions when adding functionality
    \item \textbf{State management}: Track changes and dependencies across sessions
    \item \textbf{Context}: 40K-200K tokens, extended development workflows
\end{itemize}

\subsubsection{6. Code Comprehension (1,000 instances)}
Tests deep understanding and explanation capabilities:
\begin{itemize}
    \item \textbf{Algorithm explanation}: Describe complex algorithms and their purpose
    \item \textbf{Architecture documentation}: Generate comprehensive system documentation
    \item \textbf{Code review}: Identify issues, improvements, and architectural concerns
    \item \textbf{Knowledge extraction}: Extract patterns, practices, and design principles
    \item \textbf{Context}: 10K-80K tokens, focus on understanding over generation
\end{itemize}

\subsubsection{7. Integration Testing (1,000 instances)}
Evaluates understanding of system integration and testing:
\begin{itemize}
    \item \textbf{Component interaction testing}: Verify interfaces between modules
    \item \textbf{API compatibility validation}: Ensure backward compatibility in changes
    \item \textbf{End-to-end workflow testing}: Test complete user journeys across systems
    \item \textbf{Deployment validation}: Verify changes work in production-like environments
    \item \textbf{Context}: 35K-100K tokens, system-level testing scenarios
\end{itemize}

\subsubsection{8. Security Analysis (1,000 instances)}
Tests security-aware development practices:
\begin{itemize}
    \item \textbf{Vulnerability detection}: Identify SQL injection, XSS, authentication flaws
    \item \textbf{Access control analysis}: Verify proper authorization across endpoints
    \item \textbf{Data flow security}: Trace sensitive data through system components
    \item \textbf{Secure coding practices}: Implement security patterns and best practices
    \item \textbf{Context}: 20K-80K tokens, security-critical code scenarios
\end{itemize}

\subsection{Difficulty Levels and Context Distribution}

AgentCodeEval implements a sophisticated difficulty progression based on context length, architectural complexity, and information coverage:

\begin{enumerate}
    \item \textbf{Easy Level (2,000 instances)}
    \begin{itemize}
        \item \textbf{Context Length}: 10K-40K tokens (15-30 files)
        \item \textbf{Complexity}: Basic architectural patterns, straightforward dependencies
        \item \textbf{Tasks}: Simple refactoring, basic feature addition, clear bug scenarios
        \item \textbf{Information Coverage}: IC = 0.7-0.8 (moderate context utilization)
        \item \textbf{Session Count}: 1-2 sessions for multi-session tasks
    \end{itemize}
    
    \item \textbf{Medium Level (4,000 instances)}
    \begin{itemize}
        \item \textbf{Context Length}: 40K-100K tokens (30-80 files)
        \item \textbf{Complexity}: Multiple architectural layers, moderate dependency chains
        \item \textbf{Tasks}: Cross-module refactoring, complex feature integration, multi-component debugging
        \item \textbf{Information Coverage}: IC = 0.8-0.9 (high context utilization)
        \item \textbf{Session Count}: 2-4 sessions for multi-session tasks
    \end{itemize}
    
    \item \textbf{Hard Level (2,000 instances)}
    \begin{itemize}
        \item \textbf{Context Length}: 100K-200K+ tokens (80-200+ files)
        \item \textbf{Complexity}: Complex microservice architectures, deep dependency networks
        \item \textbf{Tasks}: System-wide changes, architectural refactoring, complex integration scenarios
        \item \textbf{Information Coverage}: IC > 0.9 (very high context utilization)
        \item \textbf{Session Count}: 3-5 sessions for multi-session tasks
        \item \textbf{Expert-Level Evaluation}: Includes enterprise-scale architectures and maximum complexity scenarios
    \end{itemize}
\end{enumerate}

\subsection{Progressive Complexity Design}
Each difficulty level introduces additional challenges:
\begin{itemize}
    \item \textbf{Architectural Depth}: From single-layer to multi-tier architectures
    \item \textbf{Dependency Complexity}: From linear to circular and nested dependencies
    \item \textbf{Context Span}: From local modules to entire repository understanding
    \item \textbf{Session Memory**: From short-term to long-term context retention
\end{itemize}

\section{Data Sources and Infrastructure}

\subsection{Foundation Datasets}
\begin{table}[h]
\centering
\begin{tabular}{@{}lll@{}}
\toprule
Dataset & Size & Description \\
\midrule
The Stack v2 (BigCode) & 67.5TB & 3B+ files, 600+ languages, production codebases \\
StarCoderData & 783GB & Refined, deduplicated version with PII removal \\
Project CodeNet (IBM) & 14M samples & Problem-solution pairs across 4K problems \\
CodeSearchNet & 2M pairs & Comment-code pairs for documentation tasks \\
\bottomrule
\end{tabular}
\caption{Primary data sources for AgentCodeEval generation}
\end{table}

\subsection{Code Analysis Infrastructure}
\begin{itemize}
    \item \textbf{Tree-sitter}: Multi-language incremental parser supporting 358+ languages
    \item \textbf{rust-code-analysis (Mozilla)}: Production-grade metrics extraction and complexity analysis
    \item \textbf{metric-gardener}: Multi-language static analysis for dependency mapping
    \item \textbf{tree-climber}: Program analysis tools for control flow and data flow analysis
\end{itemize}

\subsection{Evaluation Framework Integration}
\begin{itemize}
    \item \textbf{LightEval (Hugging Face)}: Extended for agent-specific evaluation with vLLM support
    \item \textbf{LOOM-Scope}: Long-context evaluation framework integration
    \item \textbf{Custom Agent Metrics}: Novel metrics for architectural coherence and memory retention
    \item \textbf{LLM-as-a-Judge}: Automated evaluation using GPT-4o/Claude for complex reasoning assessment
\end{itemize}

\subsection{Repository Selection Criteria}
From The Stack v2's 3B+ files, repositories are selected based on:
\begin{itemize}
    \item \textbf{Quality Metrics}: >100 stars, active development, clean commit history
    \item \textbf{Architectural Complexity}: Multiple modules, clear patterns, realistic dependencies
    \item \textbf{Size Range}: 50-500 files per repository (10K-200K tokens)
    \item \textbf{Language Diversity}: Python, JavaScript, Java, Go, Rust, TypeScript
    \item \textbf{Domain Coverage}: Web frameworks, ML libraries, system tools, enterprise applications
\end{itemize}

\subsection{Programming Language Selection Rationale}

AgentCodeEval employs a **Top 10 Languages Strategy** (10 languages × 100 projects = 1,000 total projects) to maximize programming language diversity while maintaining substantial scale per language. Our selection combines global market rankings, agent development relevance, and enterprise adoption patterns based on 2025 data.

\textbf{Selected Languages (10 languages, optimized for comprehensive coverage):}

\begin{table}[h]
\centering
\begin{tabular}{|l|c|c|l|}
\hline
\textbf{Language} & \textbf{TIOBE 2025} & \textbf{Trend} & \textbf{Agent Relevance} \\
\hline
Python & \#1 (23.88\%) & +8.72\% & AI/ML, automation, data science \\
C++ & \#2 (11.37\%) & +0.84\% & Performance, games, embedded systems \\
Java & \#3 (10.66\%) & +1.79\% & Enterprise, Android, backend services \\
C & \#4 (9.84\%) & -1.14\% & Systems, OS development, embedded \\
C\# & \#5 (4.12\%) & -3.41\% & Enterprise, Windows, .NET ecosystem \\
JavaScript & \#6 (3.78\%) & +0.61\% & Web development, full-stack \\
TypeScript & JS ecosystem & Growing & Enterprise web, type safety \\
Go & \#8 (2.26\%) & +0.53\% & Cloud-native, microservices \\
Rust & \#13 (1.47\%) & +0.42\% & Systems, security, memory safety \\
PHP & \#14 (1.14\%) & -0.37\% & Web backends, legacy systems \\
\hline
\end{tabular}
\caption{AgentCodeEval 10-Language Selection (2025 TIOBE Index + Enterprise Data)}
\end{table}

\textbf{Strategic Inclusion Justifications:}
\begin{itemize}
    \item \textbf{Top 6 Global Languages}: Python, C++, Java, C, C\#, JavaScript represent 74\% of global programming activity
    \item \textbf{Enterprise Web Standard}: TypeScript adoption in 75\% of React projects, becoming enterprise JavaScript standard
    \item \textbf{Cloud-Native Leader}: Go's dominance in microservices, container orchestration (Docker, Kubernetes)
    \item \textbf{Security \& Performance}: Rust's growing adoption for memory-safe systems programming
    \item \textbf{Legacy Web Systems}: PHP still powers 77\% of websites, critical for agent migration scenarios
\end{itemize}

\textbf{Coverage Analysis:}
\begin{itemize}
    \item \textbf{Systems Programming}: C, C++, Rust (low-level, performance-critical)
    \item \textbf{Enterprise Development}: Java, C\#, TypeScript (large-scale applications)
    \item \textbf{Web Development}: JavaScript, TypeScript, PHP (full-stack scenarios)
    \item \textbf{AI/ML \& Data}: Python (machine learning, automation)
    \item \textbf{Cloud \& Infrastructure}: Go, Rust (modern distributed systems)
    \item \textbf{Mobile \& Gaming}: Java (Android), C++ (AAA games)
\end{itemize}

\textbf{Alternative Languages Considered:}

\begin{table}[h]
\centering
\begin{tabular}{|l|c|l|}
\hline
\textbf{Language} & \textbf{TIOBE 2025} & \textbf{Exclusion Rationale} \\
\hline
Swift & \#16 & iOS-specific, limited cross-platform agent scenarios \\
Kotlin & \#17 & Android-specific, overlaps with Java coverage \\
Ruby & \#19 & Declining usage, niche web development \\
Dart & \#28 & Flutter-specific, limited enterprise adoption \\
\hline
\end{tabular}
\caption{Languages Considered but Excluded}
\end{table}

\textbf{Optimized Scale Rationale:}
The **10 × 100 approach** (1,000 total projects) provides:
\begin{itemize}
    \item \textbf{Comprehensive Language Coverage}: Represents 85\%+ of global programming activity
    \item \textbf{Sufficient Sample Size}: 100 projects per language ensures statistical significance
    \item \textbf{Practical Generation Scale}: Manageable for API costs while maintaining research validity
    \item \textbf{Balanced Complexity}: Each language generates 800 scenarios (8 categories × 100 projects)
\end{itemize}

\subsubsection{Agent-Centric Selection Criteria}

The final language selection prioritizes agent development realities over pure market rankings:

\begin{itemize}
    \item \textbf{Growth Trajectory}: All selected languages show positive or stable growth trends
    \item \textbf{Agent Use Cases}: Coverage of web development, cloud infrastructure, AI/ML, and enterprise systems
    \item \textbf{Modern Safety Features}: Emphasis on memory safety (Rust), type safety (TypeScript), and robust error handling
    \item \textbf{Developer Productivity}: Rich ecosystems, strong tooling, and rapid development capabilities
    \item \textbf{Industry Future**: Alignment with cloud-native, AI-first development paradigms
\end{itemize}

This selection ensures AgentCodeEval evaluates agent capabilities in the programming languages and paradigms they will most frequently encounter in real-world 2025+ development scenarios.

\subsection{Target Repository Categories}
\begin{table}[h]
\centering
\begin{tabular}{@{}lll@{}}
\toprule
Category & Examples & Instance Count \\
\midrule
AI/ML Frameworks & TensorFlow, PyTorch, scikit-learn (Python) & 2,400 \\
Web Frameworks & React, Vue (JavaScript), Django, Flask (Python) & 2,400 \\
Cloud Services & Docker, Kubernetes (Go), FastAPI (Python) & 1,800 \\
Enterprise Apps & Spring Boot (Java), Enterprise systems & 1,800 \\
API Services & REST APIs (Go, Java), GraphQL services & 1,800 \\
System Tools & Rust-based tools, Go utilities & 1,200 \\
Security Tools & Authentication (Rust, Go), encryption & 1,200 \\
Modern Web Apps & TypeScript applications, Node.js services & 600 \\
\bottomrule
\end{tabular}
\caption{Repository distribution optimized for agent-relevant domains with selected languages}
\end{table}

\section{Methodology}

\subsection{Automated Generation Pipeline}

AgentCodeEval employs a sophisticated 4-phase automated generation pipeline to create 8,000 high-quality evaluation instances:

\subsubsection{Phase 1: Repository Analysis and Selection}
\begin{enumerate}
    \item \textbf{Repository Filtering}: Process The Stack v2's 3B+ files using quality metrics (stars >100, clean history)
    \item \textbf{AST Parsing}: Use tree-sitter to parse selected repositories into language-agnostic ASTs
    \item \textbf{Dependency Analysis}: Extract import graphs, function call chains, and module relationships
    \item \textbf{Architectural Pattern Detection}: Identify MVC, microservice, layered architectures using pattern matching
    \item \textbf{Complexity Metrics}: Calculate cyclomatic complexity, dependency depth, and cohesion metrics
    \item \textbf{Output}: 50,000 analyzed repositories with rich metadata
\end{enumerate}

\subsubsection{Phase 2: Scenario Template Generation}
\begin{enumerate}
    \item \textbf{Template Creation}: Develop 15 scenario templates per task category (120 total)
    \item \textbf{Context Window Planning}: Design progressive complexity for 10K-200K token ranges
    \item \textbf{Information Coverage Optimization**: Ensure IC > 0.7 by distributing critical information across context
    \item \textbf{Multi-session Workflow Design**: Create 2-5 session templates for development workflows
    \item \textbf{Validation Framework**: Implement automated checks for task solvability and complexity
    \item \textbf{Output**: 120 validated scenario templates with complexity calibration
\end{enumerate}

\subsubsection{Phase 3: Instance Generation at Scale}
\begin{enumerate}
    \item \textbf{Template-Repository Matching**: Pair scenarios with appropriate repositories based on complexity
    \item \textbf{Synthetic Variation Generation**: Create 100 variations per template using LLM APIs
    \item \textbf{Context Length Distribution**: Ensure proper distribution across Easy/Medium/Hard/Expert levels
    \item \textbf{Quality Assurance Pipeline**: Automated validation of syntax, dependencies, and information coverage
    \item \textbf{Deduplication**: Remove near-duplicate instances using semantic similarity
    \item \textbf{Output**: 8,000 unique, validated evaluation instances
\end{enumerate}

\subsubsection{Phase 4: Automated Validation Framework}
\begin{enumerate}
    \item \textbf{Test-Driven Validation**: Automated compilation, unit testing, and integration testing
    \item \textbf{Code Quality Analysis**: Algorithmic complexity, maintainability, and security analysis
    \item \textbf{Agent-Specific Metrics**: Implementation of ACS, DTA, MMR, CFRD, IDC, ICU calculations
    \item \textbf{Objective Evaluation Framework**: Fully automated assessment without human bias
    \item \textbf{Performance Benchmarking**: Cross-model comparison using standardized metrics
    \item \textbf{Output**: Fully automated benchmark with objective evaluation framework
\end{enumerate}

\subsection{Project Uniqueness Guarantee System}

AgentCodeEval implements a sophisticated \textbf{Multi-Layer Uniqueness System} to guarantee 100 completely unique projects per language across our 10 selected programming languages (1,000 total projects). This addresses the critical challenge of ensuring no repetitive or similar projects in the benchmark.

\subsubsection{Uniqueness Challenge Analysis}
With a target of 100 projects per language, traditional domain-complexity combinations (10 domains × 4 complexity levels = 40 combinations) would result in significant repetition. Our solution expands the uniqueness space through multiple orthogonal factors.

\subsubsection{Multi-Factor Uniqueness Architecture}

\textbf{Expanded Domain Categories (36 subcategories):}
Instead of generic categories, we implement specific project subcategories:
\begin{itemize}
    \item \textbf{Web Applications (6)}: E-commerce, social platforms, CMS, dashboards, blogs, portfolios
    \item \textbf{API Services (4)}: REST APIs, GraphQL services, microservices, API gateways
    \item \textbf{Data Systems (5)}: Analytics platforms, ETL pipelines, data warehouses, streaming, data lakes
    \item \textbf{ML/AI Systems (4)}: Training platforms, inference services, NLP systems, computer vision
    \item \textbf{Desktop Applications (3)}: Productivity tools, media applications, development environments
    \item \textbf{Mobile Applications (3)}: Social apps, utility apps, mobile games
    \item \textbf{System Infrastructure (4)}: Monitoring, automation, networking, security tools
    \item \textbf{Financial Technology (3)}: Payment systems, trading platforms, banking applications
    \item \textbf{Gaming \& Simulation (2)}: Game engines, simulation frameworks
    \item \textbf{Blockchain Systems (2)}: DeFi platforms, NFT marketplaces
\end{itemize}

\textbf{Architecture Pattern Variations (10 types):}
\begin{itemize}
    \item Monolithic, Microservices, Serverless, Event-Driven
    \item Layered, Clean Architecture, Hexagonal, MVC, MVVM, Component-Based
\end{itemize}

\textbf{Project Theme Classifications (8 categories):}
\begin{itemize}
    \item Business, Education, Healthcare, Entertainment
    \item Productivity, Social, Utility, Creative
\end{itemize}

\textbf{Deterministic Seed Generation:}
Each project receives a unique deterministic seed based on:
\texttt{hash(language-domain-complexity-architecture-theme-index) \% 1,000,000}

\subsubsection{Mathematical Uniqueness Guarantee}

\begin{table}[h]
\centering
\begin{tabular}{@{}lr@{}}
\toprule
\textbf{Uniqueness Factor} & \textbf{Count} \\
\midrule
Domain Subcategories & 36 \\
Complexity Levels & 4 \\
Architecture Patterns & 10 \\
Project Themes & 8 \\
Unique Seeds & ∞ \\
\midrule
\textbf{Total Combinations} & \textbf{11,520} \\
\textbf{Target per Language} & \textbf{100} \\
\textbf{Uniqueness Ratio} & \textbf{115.2×} \\
\bottomrule
\end{tabular}
\caption{Project Uniqueness Factor Analysis}
\end{table}

\textbf{Calculation:}
36 domains × 4 complexities × 10 architectures × 8 themes = 11,520 unique combinations

\textbf{Result:} Each project gets a completely unique combination of factors, with 115× more combinations available than needed.

\subsubsection{Uniqueness Implementation Strategy}

\textbf{Distributed Selection Algorithm:}
\begin{enumerate}
    \item Cycle through domains systematically (domains[i \% 36])
    \item Apply complexity distribution from configuration
    \item Rotate architecture patterns (architectures[i \% 10])
    \item Cycle through themes (themes[i \% 8])
    \item Generate deterministic but unique seeds for LLM variation
\end{enumerate}

\textbf{LLM Prompt Specialization:}
Each project receives tailored prompts emphasizing:
\begin{itemize}
    \item Domain-specific requirements and constraints
    \item Architecture pattern implementation details
    \item Theme-appropriate features and use cases
    \item Complexity-appropriate scope and technical depth
\end{itemize}

\textbf{Verification and Quality Control:}
\begin{itemize}
    \item Automated uniqueness validation of project IDs
    \item Cross-verification of factor combinations
    \item Deterministic reproducibility through seed management
    \item Quality assurance of generated project specifications
\end{itemize}

\subsubsection{Benefits of Multi-Layer Uniqueness}

\begin{enumerate}
    \item \textbf{Complete Elimination of Repetition}: Mathematical guarantee of unique projects
    \item \textbf{Realistic Diversity}: Each project addresses distinct real-world scenarios
    \item \textbf{Comprehensive Coverage}: Spans entire spectrum of software development domains
    \item \textbf{Deterministic Generation}: Same configuration always produces identical benchmark
    \item \textbf{Scalable Architecture}: Can support 1,000+ projects per language if needed
    \item \textbf{LLM Variation Control}: Seeds ensure different implementations for same factor combinations
\end{enumerate}

\subsubsection{Sample Unique Project Examples}

\textbf{Python Projects:}
\begin{itemize}
    \item \texttt{python\_web\_ecommerce\_hard\_001}: Healthcare e-commerce platform with microservices architecture
    \item \texttt{python\_api\_rest\_medium\_002}: Education REST API with clean architecture pattern
    \item \texttt{python\_ml\_nlp\_expert\_003}: Business NLP platform using event-driven design
    \item \texttt{python\_data\_analytics\_easy\_004}: Social analytics dashboard with MVC architecture
\end{itemize}

\textbf{JavaScript Projects:}
\begin{itemize}
    \item \texttt{javascript\_web\_social\_medium\_001}: Entertainment social platform with component-based architecture
    \item \texttt{javascript\_mobile\_utility\_hard\_002}: Productivity mobile app using serverless architecture
\end{itemize}

This uniqueness system ensures that AgentCodeEval provides genuinely diverse evaluation scenarios, eliminating concerns about repetitive or similar projects across the 1,000-project benchmark.

\subsection{Novel Agent-Specific Evaluation Metrics}

AgentCodeEval introduces comprehensive evaluation metrics specifically designed for software development agents:

\subsubsection{Architectural Coherence Score (ACS)}
Measures consistency with existing architectural patterns and design principles:
\begin{itemize}
    \item \textbf{Pattern Adherence}: Consistency with MVC, Observer, Factory patterns (0-1 scale)
    \item \textbf{Layer Separation**: Proper separation between presentation, business, data layers
    \item \textbf{Module Cohesion**: Logical grouping of related functionality
    \item \textbf{Interface Consistency**: Adherence to established API contracts
    \item \textbf{Formula}: ACS = 0.3×Pattern + 0.3×Separation + 0.2×Cohesion + 0.2×Interface
\end{itemize}

\subsubsection{Dependency Traversal Accuracy (DTA)}
Evaluates correct navigation and understanding of complex dependency networks:
\begin{itemize}
    \item \textbf{Import Resolution**: Correct identification of required dependencies
    \item \textbf{Call Chain Accuracy**: Proper understanding of function call sequences
    \item \textbf{Circular Dependency Detection**: Identification and handling of dependency cycles
    \item \textbf{Abstraction Level Respect**: Maintaining proper architectural boundaries
    \item \textbf{Formula}: DTA = 0.4×Import + 0.3×CallChain + 0.15×Circular + 0.15×Abstraction
\end{itemize}

\subsubsection{Multi-Session Memory Retention (MMR)}
Novel metric for evaluating context persistence across development sessions:
\begin{itemize}
    \item \textbf{Decision Consistency**: Maintaining architectural decisions across sessions
    \item \textbf{Context Carryover**: Remembering constraints and requirements from previous sessions
    \item \textbf{Incremental Building**: Successfully building on previous implementations
    \item \textbf{State Tracking**: Maintaining awareness of changes and their implications
    \item \textbf{Formula**: MMR = 0.3×Decision + 0.3×Context + 0.25×Building + 0.15×State
\end{itemize}

\subsubsection{Cross-File Reasoning Depth (CFRD)}
Measures understanding of relationships across multiple files:
\begin{itemize}
    \item \textbf{Inter-module Understanding**: Comprehension of cross-file relationships
    \item \textbf{Data Flow Tracking**: Following data transformations across modules
    \item \textbf{Control Flow Analysis**: Understanding execution paths across files
    \item \textbf{Side Effect Recognition**: Identifying cross-module side effects
    \item \textbf{Formula**: CFRD = 0.3×Inter + 0.25×DataFlow + 0.25×ControlFlow + 0.2×SideEffect
\end{itemize}

\subsubsection{Incremental Development Capability (IDC)}
Evaluates ability to build upon existing implementations progressively:
\begin{itemize}
    \item \textbf{Feature Extension**: Successfully adding functionality to existing systems
    \item \textbf{Refactoring Preservation**: Maintaining functionality during restructuring
    \item \textbf{Integration Compatibility**: Ensuring new features work with existing components
    \item \textbf{Regression Prevention**: Avoiding introduction of bugs in existing functionality
    \item \textbf{Formula**: IDC = 0.3×Extension + 0.25×Preservation + 0.25×Integration + 0.2×Prevention
\end{itemize}

\subsubsection{Information Coverage Utilization (ICU)}
Measures how effectively agents use the provided context:
\begin{itemize}
    \item \textbf{Context Span**: Percentage of provided context actually referenced
    \item \textbf{Critical Information Identification**: Recognition of key architectural components
    \item \textbf{Relevance Filtering**: Ability to distinguish relevant from irrelevant information
    \item \textbf{Deep vs Surface Understanding**: Quality of context utilization
    \item \textbf{Formula**: ICU = 0.3×Span + 0.3×Critical + 0.2×Filtering + 0.2×Depth
\end{itemize}

\subsubsection{Composite Agent Development Score (CADS)}
Overall agent capability combining all metrics:
\begin{itemize}
    \item \textbf{CADS = 0.2×ACS + 0.2×DTA + 0.2×MMR + 0.15×CFRD + 0.15×IDC + 0.1×ICU}
    \item \textbf{Range**: 0-5 scale for consistency with existing benchmarks
    \item \textbf{Interpretation**: >4.0 (Excellent), 3.0-4.0 (Good), 2.0-3.0 (Fair), <2.0 (Poor)
\end{itemize}

\section{Achieving 8,000 Instance Scale}

\subsection{Instance Distribution Strategy}
AgentCodeEval's 8,000 instances are strategically distributed to ensure comprehensive coverage:

\begin{table}[h]
\centering
\begin{tabular}{@{}lrrrr@{}}
\toprule
Task Category & Easy & Medium & Hard & Total \\
\midrule
Architectural Understanding & 250 & 500 & 250 & 1,000 \\
Cross-file Refactoring & 250 & 500 & 250 & 1,000 \\
Feature Implementation & 250 & 500 & 250 & 1,000 \\
Bug Investigation & 250 & 500 & 250 & 1,000 \\
Multi-session Development & 250 & 500 & 250 & 1,000 \\
Code Comprehension & 250 & 500 & 250 & 1,000 \\
Integration Testing & 250 & 500 & 250 & 1,000 \\
Security Analysis & 250 & 500 & 250 & 1,000 \\
\midrule
\textbf{Total} & \textbf{2,000} & \textbf{4,000} & \textbf{2,000} & \textbf{8,000} \\
\bottomrule
\end{tabular}
\caption{AgentCodeEval instance distribution across categories and difficulty levels}
\end{table}

\subsection{Context Length Distribution}
\begin{itemize}
    \item \textbf{Short Context (10K-40K tokens)}: 2,000 instances (Easy tasks)
    \item \textbf{Medium Context (40K-100K tokens)}: 4,000 instances (Medium tasks)
    \item \textbf{Long Context (100K-200K tokens)}: 2,000 instances (Hard tasks)
    \item \textbf{Ultra-Long Context (200K+ tokens)}: Expert-level evaluation integrated across categories
\end{itemize}

\subsection{Scalability Advantages}

\subsubsection{Automated Generation Pipeline}
Our scale is achievable through automated generation using The Stack v2:
\begin{itemize}
    \item \textbf{Repository Pool}: 50,000+ high-quality repositories
    \item \textbf{Synthetic Scenario Generation}: LLM-generated realistic development scenarios
    \item \textbf{Template-based Scaling}: 8 task categories × 10 templates = 80 base patterns
    \item \textbf{Variation Generation}: 100 variations per base pattern = 8,000 instances
\end{itemize}

\subsubsection{Quality Control at Scale}
\begin{itemize}
    \item \textbf{Automated Validation}: Tree-sitter parsing ensures syntactic correctness
    \item \textbf{Complexity Verification}: Information Coverage (IC) > 0.7 for all instances
    \item \textbf{Algorithmic Validation}: Comprehensive automated quality assurance using 6 novel metrics
    \item \textbf{Difficulty Calibration}: Progressive complexity scaling within categories
\end{itemize}

\subsubsection{Resource Efficiency}
Despite 8,000 instances, resource requirements remain manageable:
\begin{itemize}
    \item \textbf{Generation Cost}: \$1,500 in API calls for synthetic project and scenario generation
    \item \textbf{Storage Requirements}: 300GB compressed benchmark data
    \item \textbf{Evaluation Time}: Fully automated evaluation with no manual intervention
    \item \textbf{No Human Annotation}: Fully automated generation and validation - eliminates human bias and subjectivity
\end{itemize}

\section{Implementation Plan}

\subsection{Phase 1: Infrastructure and Foundation (Weeks 1-2)}
\begin{itemize}
    \item \textbf{Data Access Setup}: Configure API access to The Stack v2 and StarCoderData
    \item \textbf{Analysis Pipeline**: Deploy tree-sitter, rust-code-analysis, and metric-gardener
    \item \textbf{Repository Selection**: Filter and select 50,000 high-quality repositories from The Stack v2
    \item \textbf{AST Processing Pipeline**: Implement automated parsing for Python, JavaScript, Java, C++, Go
    \item \textbf{Quality Metrics Framework**: Establish repository quality scoring (stars, complexity, patterns)
    \item \textbf{Deliverable**: 50,000 analyzed repositories with rich architectural metadata
\end{itemize}

\subsection{Phase 2: Template and Scenario Design (Weeks 3-4)}
\begin{itemize}
    \item \textbf{Task Category Templates**: Create 15 scenario templates per category (120 total)
    \item \textbf{Complexity Calibration**: Design progressive difficulty scaling for context lengths
    \item \textbf{Information Coverage Optimization**: Implement ETHIC-style IC > 0.7 requirement
    \item \textbf{Multi-session Framework**: Design 2-5 session development workflow templates
    \item \textbf{Validation Pipeline**: Automated solvability and complexity verification
    \item \textbf{Deliverable**: 120 validated scenario templates with complexity calibration
\end{itemize}

\subsection{Phase 3: Large-Scale Instance Generation (Weeks 5-6)}
\begin{itemize}
    \item \textbf{Template-Repository Matching**: Intelligent pairing based on complexity and domain
    \item \textbf{Automated Variation Generation**: LLM APIs (GPT-4o, Claude 3.5) for 100 variations per template
    \item \textbf{Quality Assurance Pipeline**: Syntax validation, dependency checking, IC verification
    \item \textbf{Deduplication System**: Semantic similarity-based duplicate removal
    \item \textbf{Distribution Validation**: Ensure proper Easy/Medium/Hard/Expert distribution
    \item \textbf{Deliverable**: 8,000 unique, validated evaluation instances
\end{itemize}

\subsection{Phase 4: Automated Validation Framework (Weeks 7-8)}
\begin{itemize}
    \item \textbf{Test-Driven Validation**: Automated compilation, unit testing, and integration testing
    \item \textbf{Code Quality Analysis**: Algorithmic complexity, maintainability, and security analysis
    \item \textbf{Agent-Specific Metrics**: Implementation of ACS, DTA, MMR, CFRD, IDC, ICU calculations
    \item \textbf{Objective Evaluation Framework**: Fully automated assessment without human bias
    \item \textbf{Performance Benchmarking**: Cross-model comparison using standardized metrics
    \item \textbf{Output**: Fully automated benchmark with objective evaluation framework
\end{itemize}

\subsection{Phase 5: Validation, Documentation and Release (Weeks 9-10)}
\begin{itemize}
    \item \textbf{Automated Validation Framework**: Comprehensive testing using 6 algorithmic metrics across all instances
    \item \textbf{Benchmark Reliability Analysis**: Statistical consistency testing, metric validation
    \item \textbf{Comprehensive Documentation**: Technical documentation, usage guides, evaluation protocols
    \item \textbf{Research Paper Preparation**: Write and submit to top-tier venue (ICLR/NeurIPS/ICML)
    \item \textbf{Public Release**: Open-source benchmark, leaderboard, and evaluation tools
    \item \textbf{Deliverable**: Published benchmark, research paper, and public evaluation platform
\end{itemize}

\subsection{Success Metrics and Milestones}
\begin{itemize}
    \item \textbf{Scale Achievement**: 8,000 instances (3.5x larger than SWE-Bench)
    \item \textbf{Quality Validation**: IC > 0.7 for all instances, comprehensive automated metric validation
    \item \textbf{Model Discrimination**: Clear performance differences across difficulty levels
    \item \textbf{Statistical Significance**: Robust performance differences between model categories
    \item \textbf{Community Adoption**: >50 model submissions within 6 months of release
\end{itemize}

\section{Expected Contributions and Impact}

\subsection{Primary Research Contributions}
\begin{enumerate}
    \item \textbf{Unprecedented Scale Benchmark}: First 8,000-instance benchmark for software development agents—3.5x larger than existing code benchmarks
    \item \textbf{Novel Agent-Specific Evaluation}: Introduction of 6 new metrics (ACS, DTA, MMR, CFRD, IDC, ICU) specifically designed for development agents
    \item \textbf{Multi-Session Evaluation Paradigm}: First benchmark to evaluate context persistence and incremental development across multiple sessions
    \item \textbf{Long-Context Code Integration**: Comprehensive evaluation of 10K-200K token codebases with high information coverage (IC > 0.7)
    \item \textbf{Automated Generation Pipeline**: Scalable methodology for creating realistic development scenarios using production codebases
    \item \textbf{Architectural Understanding Assessment}: First systematic evaluation of design pattern recognition and architectural comprehension at scale
\end{enumerate}

\subsection{Technical Innovations}
\begin{enumerate}
    \item \textbf{Information Coverage for Code**: Extension of ETHIC methodology to software development with distributed dependencies
    \item \textbf{Production Codebase Integration**: Utilization of The Stack v2's 3B+ files for authentic development scenarios
    \item \textbf{Automated Validation Framework**: Objective evaluation using algorithmic metrics and test-driven validation
    \item \textbf{Progressive Complexity Scaling**: Sophisticated difficulty progression from 10K to 200K+ tokens with calibrated complexity
    \item \textbf{Agent Workflow Simulation**: Realistic multi-step development processes spanning architecture, implementation, debugging, and maintenance
\end{enumerate}

\subsection{Practical Impact}
\begin{enumerate}
    \item \textbf{Industry Standard Establishment**: Potential to become the definitive benchmark for software development agents (like HumanEval for code completion)
    \item \textbf{Model Development Guidance**: Clear metrics for improving long-context architectural understanding and memory retention
    \item \textbf{Research Acceleration**: Large-scale dataset enabling statistical analysis of agent capabilities across multiple dimensions
    \item \textbf{Commercial Development Support**: Rigorous evaluation framework for AI-powered development tools and coding assistants
    \item \textbf{Educational Applications**: Training and assessment framework for software engineering education
    \item \textbf{Open Science Advancement**: Fully open-source benchmark, evaluation tools, and comprehensive documentation
\end{enumerate}

\subsection{Long-term Research Implications}
\begin{enumerate}
    \item \textbf{Long-Context Model Development**: Drive improvements in architectural understanding and memory retention for long-context LLMs
    \item \textbf{Agent Architecture Research}: Enable systematic study of multi-session development workflows and context persistence
    \item \textbf{Software Engineering AI**: Establish foundation for AI-assisted software development research and practice
    \item \textbf{Benchmark Methodology**: Demonstrate scalable approaches for creating large-scale, high-quality evaluation datasets
    \item \textbf{Interdisciplinary Impact**: Bridge software engineering and AI research communities through shared evaluation standards
\end{enumerate}

\subsection{Expected Community Adoption}
\begin{itemize}
    \item \textbf{Academic Research**: >100 research papers citing AgentCodeEval within 2 years
    \item \textbf{Industry Adoption**: Integration into major AI development platforms and evaluation suites
    \item \textbf{Model Benchmarking**: Standard evaluation for all major LLM releases in software development domain
    \item \textbf{Educational Integration**: Adoption in computer science curricula for AI and software engineering courses
    \item \textbf{Open Source Development**: Active community contributions and extensions to the benchmark framework
\end{itemize}

\section{Resource Requirements}

\subsection{Computational Resources}
\begin{itemize}
    \item Standard server with 64GB+ RAM (no GPU required for analysis)
    \item 300GB+ storage for datasets and 8,000 generated tasks
    \item API access to Claude/GPT-4/Gemini for synthetic generation
    \item Distributed API key management for parallel generation (estimated \$1,500 total cost)
\end{itemize}

\subsection{No Requirements}
\begin{itemize}
    \item Human annotation or labeling
    \item GPU resources for training
    \item Complex distributed infrastructure
\end{itemize}

\section{Timeline and Deliverables}

\subsection{10-Week Timeline}
\begin{itemize}
    \item Weeks 1-2: Foundation setup and tool configuration
    \item Weeks 3-4: Task generation pipeline development
    \item Weeks 5-6: Reference solution creation and validation
    \item Weeks 7-8: Benchmark assembly and initial evaluation
    \item Weeks 9-10: Validation, documentation, and publication
\end{itemize}

\subsection{Benchmark Scale Comparison}
\begin{table}[h]
\centering
\begin{tabular}{@{}lrr@{}}
\toprule
Benchmark & Instances & Context Length \\
\midrule
HumanEval & 164 & Single function \\
MBPP & 1,000 & Single function \\
SWE-Bench & 2,294 & Repository-level \\
L-Eval & 2,000+ & 3K-200K tokens \\
ETHIC & 1,986 & High coverage \\
AgentRewardBench & 1,302 & Web trajectories \\
\textbf{AgentCodeEval} & \textbf{8,000} & \textbf{10K-200K tokens} \\
\bottomrule
\end{tabular}
\caption{Benchmark scale comparison - AgentCodeEval provides 3.5x more instances than the largest existing benchmark}
\end{table}

\subsection{Key Deliverables}
\begin{enumerate}
    \item AgentCodeEval benchmark dataset (8,000 instances)
    \item Automated evaluation framework and metrics
    \item Baseline performance analysis of major LLMs
    \item Research paper documenting methodology and results
    \item Open-source release of tools and benchmark
\end{enumerate}

\section{Conclusion}

AgentCodeEval addresses a critical gap in long-context LLM evaluation by focusing on realistic software development agent tasks. With 8,000 instances—3.5x larger than the current largest benchmark—AgentCodeEval provides unprecedented comprehensive coverage of agent-based software development scenarios. By leveraging existing high-quality code datasets and automated analysis tools, we can create this large-scale benchmark without requiring extensive human annotation or GPU resources.

The proposed approach is both novel and practical, offering a scalable methodology for generating realistic evaluation scenarios while maintaining rigorous evaluation standards. The 8,000-instance scale enables statistically significant analysis across multiple dimensions: context length, task complexity, programming languages, and architectural patterns. AgentCodeEval has the potential to become the definitive standard benchmark in the field, establishing new baselines for long-context software development agent evaluation.

\section{AgentCodeEval vs. Existing Benchmarks: Key Differentiators}

\subsection{Unique Positioning}
AgentCodeEval distinguishes itself from existing evaluation frameworks through several key innovations:

\begin{table}[h]
\centering
\begin{tabular}{@{}lllll@{}}
\toprule
Benchmark & Context Length & Task Type & Agent Focus & Evaluation Scope \\
\midrule
HumanEval & Single function & Code completion & None & Function-level \\
MBPP & Single function & Code generation & None & Problem-solving \\
SWE-Bench & Repository & Single issue & Limited & Issue resolution \\
L-Eval & 3K-200K tokens & General long-context & None & Document QA \\
∞Bench & 100K+ tokens & General reasoning & None & Long context \\
ETHIC & High coverage & Domain tasks & None & Information utilization \\
AgentRewardBench & Web trajectories & Web agents & Web-specific & Trajectory evaluation \\
\textbf{AgentCodeEval} & \textbf{10K-200K tokens} & \textbf{Multi-file development} & \textbf{Code agents} & \textbf{Agent workflows} \\
\bottomrule
\end{tabular}
\caption{Comparison of AgentCodeEval with existing benchmarks}
\end{table}

\subsection{Novel Evaluation Dimensions}

\subsubsection{Multi-Session Development Workflows}
Unlike existing benchmarks that evaluate single-turn interactions, AgentCodeEval introduces \textbf{multi-session evaluation}:
\begin{itemize}
    \item Session 1: Initial implementation and architecture setup
    \item Session 2: Feature addition and integration
    \item Session 3: Bug fixing and optimization
    \item Session 4: Documentation and testing
    \item Session 5: Refactoring and maintenance
\end{itemize}

\subsubsection{Architectural Understanding at Scale}
AgentCodeEval is the first benchmark to systematically evaluate \textbf{architectural comprehension}:
\begin{itemize}
    \item Design pattern recognition across 50+ files
    \item Dependency graph understanding with 100+ components
    \item Microservice architecture navigation
    \item Database schema integration reasoning
\end{itemize}

\subsubsection{Long-Context Agent Memory}
Novel evaluation of \textbf{context persistence} and \textbf{incremental development}:
\begin{itemize}
    \item Maintaining architectural decisions across sessions
    \item Building upon previous implementation choices
    \item Context-aware code generation that respects existing patterns
    \item Memory of cross-file dependencies and constraints
\end{itemize}

\subsection{Technical Innovation}

\subsubsection{High Information Coverage for Code}
Extending ETHIC's Information Coverage (IC) methodology to software development:
\begin{itemize}
    \item Code IC > 0.7: Tasks require 70%+ of provided codebase
    \item Distributed dependencies: No single file contains the solution
    \item Cross-module reasoning: Solutions span multiple architectural layers
\end{itemize}

\subsubsection{Agent-Specific Evaluation Metrics}
Novel metrics designed specifically for code agent evaluation:
\begin{itemize}
    \item \textbf{Architectural Coherence Score (ACS)}: Measures consistency with existing patterns
    \item \textbf{Dependency Traversal Accuracy (DTA)}: Correct navigation of complex dependencies
    \item \textbf{Multi-Session Memory Retention (MMR)}: Context maintenance across sessions
    \item \textbf{Cross-File Reasoning Depth (CFRD)}: Understanding multi-file relationships
    \item \textbf{Incremental Development Capability (IDC)}: Building on previous work
\end{itemize}

\subsubsection{Real-World Codebase Complexity}
AgentCodeEval uses actual production codebases from The Stack v2:
\begin{itemize}
    \item Real architectural patterns from 1000+ repositories
    \item Authentic dependency structures and complexity
    \item Production-level code quality and conventions
    \item Diverse programming paradigms and languages
\end{itemize}

\end{document}
